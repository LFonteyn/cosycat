% This is "sig-alternate.tex" V2.0 May 2012
% This file should be compiled with V2.5 of "sig-alternate.cls" May 2012
%
% This example file demonstrates the use of the 'sig-alternate.cls'
% V2.5 LaTeX2e document class file. It is for those submitting
% articles to ACM Conference Proceedings WHO DO NOT WISH TO
% STRICTLY ADHERE TO THE SIGS (PUBS-BOARD-ENDORSED) STYLE.
% The 'sig-alternate.cls' file will produce a similar-looking,
% albeit, 'tighter' paper resulting in, invariably, fewer pages.
%
% ----------------------------------------------------------------------------------------------------------------
% This .tex file (and associated .cls V2.5) produces:
%       1) The Permission Statement
%       2) The Conference (location) Info information
%       3) The Copyright Line with ACM data
%       4) NO page numbers
%
% as against the acm_proc_article-sp.cls file which
% DOES NOT produce 1) thru' 3) above.
%
% Using 'sig-alternate.cls' you have control, however, from within
% the source .tex file, over both the CopyrightYear
% (defaulted to 200X) and the ACM Copyright Data
% (defaulted to X-XXXXX-XX-X/XX/XX).
% e.g.
%% \CopyrightYear{2016} will cause 2007 to appear in the copyright line.
% \crdata{0-12345-67-8/90/12} will cause 0-12345-67-8/90/12 to appear in the copyright line.
%
% ---------------------------------------------------------------------------------------------------------------
% This .tex source is an example which *does* use
% the .bib file (from which the .bbl file % is produced).
% REMEMBER HOWEVER: After having produced the .bbl file,
% and prior to final submission, you *NEED* to 'insert'
% your .bbl file into your source .tex file so as to provide
% ONE 'self-contained' source file.
%
% ================= IF YOU HAVE QUESTIONS =======================
% Questions regarding the SIGS styles, SIGS policies and
% procedures, Conferences etc. should be sent to
% Adrienne Griscti (griscti@acm.org)
%
% Technical questions _only_ to
% Gerald Murray (murray@hq.acm.org)
% ===============================================================
%
% For tracking purposes - this is V2.0 - May 2012

\documentclass{sig-alternate}

\begin{document}
%
% --- Author Metadata here ---
\conferenceinfo{DATECH}{2016 Pozna\'n, Poland}
%\CopyrightYear{2007} % Allows default copyright year (20XX) to be over-ridden - IF NEED BE.
%\crdata{0-12345-67-8/90/01}  % Allows default copyright data (0-89791-88-6/97/05) to be over-ridden - IF NEED BE.
% --- End of Author Metadata ---

\title{
  Enabling Annotation of Historical Corpora in an Asynchronous Collaborative Environment
}
\subtitle{
  [Extended Abstract]
  \titlenote{
    This research was supported by ...
  }  
}

%
% You need the command \numberofauthors to handle the 'placement
% and alignment' of the authors beneath the title.
%
% For aesthetic reasons, we recommend 'three authors at a time'
% i.e. three 'name/affiliation blocks' be placed beneath the title.
%
% NOTE: You are NOT restricted in how many 'rows' of
% "name/affiliations" may appear. We just ask that you restrict
% the number of 'columns' to three.
%
% Because of the available 'opening page real-estate'
% we ask you to refrain from putting more than six authors
% (two rows with three columns) beneath the article title.
% More than six makes the first-page appear very cluttered indeed.
%
% Use the \alignauthor commands to handle the names
% and affiliations for an 'aesthetic maximum' of six authors.
% Add names, affiliations, addresses for
% the seventh etc. author(s) as the argument for the
% \additionalauthors command.
% These 'additional authors' will be output/set for you
% without further effort on your part as the last section in
% the body of your article BEFORE References or any Appendices.

\numberofauthors{2} %  in this sample file, there are a *total*
% of EIGHT authors. SIX appear on the 'first-page' (for formatting
% reasons) and the remaining two appear in the \additionalauthors section.
%
\author{
% You can go ahead and credit any number of authors here,
% e.g. one 'row of three' or two rows (consisting of one row of three
% and a second row of one, two or three).
%
% The command \alignauthor (no curly braces needed) should
% precede each author name, affiliation/snail-mail address and
% e-mail address. Additionally, tag each line of
% affiliation/address with \affaddr, and tag the
% e-mail address with \email.
%
% 1st. author
\alignauthor
Enrique Manjavacas Ar\'evalo\\
\affaddr{Universiteit Antwerpen}\\
\affaddr{R.112, Rodestraat 14}\\
\affaddr{Antwerp 2000, Belgium}\\
\email{enrique.manjavacas@uantwerpen.be}
% 2nd. author
\alignauthor
Peter Petr\'e\\
\affaddr{Universiteit Antwerpen}\\
\affaddr{R.229, Rodestraat 14}\\
\affaddr{Antwerp 2000, Belgium}\\
\email{peter.petre@uantwerpen.be}
}

\maketitle
\begin{abstract}
  Current research in Corpus Linguistics and other related disciplines within the
  multi-disciplanary field of Digital Humanities, involves computer-aided manual processing
  of large text corpora. Typically, corpus instances are retrieved with the help of
  concoordancers or textual search engines and subsequently labeled by hand before being
  submitted to quantitative analysis.
  While well-established resources already cover the data retrieval aspect,
  less attention has been paid to the annotation process in terms of both software
  facilities and best practices. However, with the current increase in size and scope
  of research projects we envisage new needs for coordinating and managing interdependent
  analysis (meta-)data created being simultaneously created by different reasearchers.
  Current ad-hoc solutions typically involve general-purpose RTE (Real-Time Editing)
  software, whose functionality may be suboptimal for the present needs, as we shall argue.
  In the present paper we will discuss potential problems related to coordinating
  annotations in large-scale projects as well as the potential benefits that can be
  derived from an application-based approach to annotation data management.
  Furthermore, we discuss some of the design decisions that arise in the process of
  developing a principled solution.
  %% Furthermore, we sketch an application design as an attempt to provide a principled solution.
  Finally, we showcase our current implementation of a collaborative asynchronous system
  in the context of a Historical Linguistics research project involving several parallel
  studies and multiple researchers.
\end{abstract}

%% A category with the (minimum) three required fields
%% https://cran.r-project.org/web/classifications/ACM.html
\category{J.5}{Computer Applications}{Arts and Humanities}
\category{I.7.1}{Document and Text Processing}{Document and Text Editing}
%% \category{I.6.5}{Simulation and Modelling}{Model Development}

\keywords{Corpus Annotations, Asynchronous Collaboration, Groupware}

\section{Introduction}\label{sec:intro}
Current corpus-based research practices entail interacting with a single ground-truth
collection of documents \textemdash i.e. a corpus \textemdash, to which analysis metadata,
such as annotations, are added.

In a context in which the responsability for the design of the research is not shared
amongst several researchers, decisions as to how to create, design and manage annotation
metadata are rather unproblematic. However, with the current increase in size and scope of
research projects in the Digital Humanities?, we expect new issues to arise from the need
for synchronization of concurrent research activities which could otherwise lead to
inconsistency in the annotation metadata. In general, inconsistency may arise for reasons
that are both intrinsic and extrinsic to the proper research activity \textemdash i.e. having
to do, respectively, with interpretational disagreements as to the content and scope of
annotations, and problems related to concurrent modification of the annotation metadata
by several researchers. In particular, we are confronted with the following problems:
\begin{itemize}
\item Unifying label conventions and tagging systems, which may not be given as closed
  set tagging systems prior to research or may subject to revision renegotiation.
\item Synchronizing the introduction and removal of labels into dynamic tag systems,
  guaranteeing that these updates do not lead to an inconsistent tag system.
\item Ensuring that every involved researcher is notified of parallel work by other members
  of her team, allowing for in-place discussion of issues interferring with their own analyses.
\item Allowing changes to existing annotations to be backtraceable and documented, providing
  a means for resetting to previous versions and for logging the motivations
  that led to the change (in the form of change logs, threaded discussions etc.).
\end{itemize}

Furthermore, a principled solution to collaborative corpus-based research is not only
desirable for the sake of the aforementioned problems, but it also provides a number of
associated advantages which have the potential to improve the quality of the research.
For instance, managing annotation metadata through a specialized application implies
having available these metadata in a unified and structured shape, easily allowing
for the following features:

\begin{itemize}
\item Easy export and re-formatting functionality.
\item Storage of metadata in a database or indexing through a search engine.
\item Advance querying of the metadata.
\item Visualizing the metadata both as time snapshot and evolution over time.
\item Enable meta-research on research practicies thanks to the availability of structured
  metadata derived from the researcher activities (annotation updates, user behaviour, etc.).
\end{itemize}

Although to our knowledge no prior studies have addressed these issues, both in terms of
established practices and available tools\footnote{
  [TODO: look for (tentative) studies in this direction.]
}, it can be assumed that a large part of current research requiring support for a
collaborative environment typically resorts to general-purpose Real-Time document Editing
(RTE) software in the best case\footnote{
  See, for instance, \cite{Rowlands2011} and \cite{Wood2011}, for examples of reports on
  using Google Docs for research.
}, or to cloud-storage services \textemdash such as Dropbox \cite{Dropboxa} \textemdash,
which provide versioning but do not support proper collaborative editing of documents,
as specified, for instance, in \cite{Dropbox}.

While RTE software solutions represent powerful and thoroughly tested tools, there are also
a number of drawbacks associated with their use in research.
For instance, their functionality may still be too general and overshoot the specific needs of
researchers, more specialized than those of common end-users. In particular, it is usually
cumbersome to cross-reference newly added annotations with the corpus positions and corpus
context in their scope.
Furthermore, the fact that data is managed in servers owned by the service providers may not
always be compatible with legal constraints around corpus data.
Finally, RTE software solutions tend to prioritize user experience, which means that
responsive and non-blocking Graphic User Interfaces (GUIs) are privileged in detriment of
other features, such as support for document versioning, which stand in a direct trade-off
with user experience.

In the present paper we sketch an application-based approach addressing the aforemention
issues in a way that fits better the concrete needs of the practicioner.
More concretely, we propose to address the problem of managing concurrent annotation
metadata by multiple researchers in the terms of collaborative working systems.
Therefore, in sections \ref{subsec:rte} and \ref{subsec:vcs}, we will review two major and
not necessarily competing approaches to concurrent multi-user document editing
(Real-time document Editing systems and Version Control Systems).
Next, in section \ref{subsec:hybrid}, we argue that a hybrid approach may be better suit for
the issues established in the current section.
In section \ref{sec:case}, we describe the research infrastructure that we have implemented
following the design principles stated in section \ref{sec:cde}.
Finally, in section \ref{sec:future} we shall highlight lines of further research before
concluding in \ref{sec:conclusion}.

\section{Collaborative Editing}\label{sec:cde}
Collaborative editing refers to the practice by which dispersed
users are able to concurrently modify shared artifacts with the guarantee that changes
by different users will not overwrite each other \textemdash see also \cite{Altmanninger2009}.
Currently, collaborative software has become ubiquitous thanks to Web 2.0 services, such as
Google's web-based suite Google Docs \cite{Googlea} or
Microsoft's Office Online \cite{Microsoft}.

We note, however, that other approaches are available beyond the RTE framework. In
particular, Version Control Systems (VCS) \textemdash also known as Revision Control
Systems (RCS), or Source Code Manager (SCM) due to their popularity among software
developers \textemdash, represent a solution to collaborative editing that is often
overlooked due arguably to its less user-friendly nature and more complex document
updates procedures.

In general, RTE and VCS can be seen as alternative solutions to collaborative
editing \cite{Altmanninger2009}, differring with respect to, respectively, whether edit
operations are non-blocking synchronous operations, and immediatly reflected on the shared
data, or asynchronous \textemdash i.e. updates are first piped
through a working copy which is only sporadically merged with the source document.
In the next two sections (\ref{subsec:rte} and \ref{subsec:vcs}) we summarily describe RTE
and VCS systems.

\subsection{Real-Time document Editing}\label{subsec:rte}
RTE software system aim at solving the synchronous flavour of the collaborative editing
problem, which involves the following two-step procedure \cite{Imine2009}: first, local
changes are immeditaely reflected in the local copy of the document and, secondly,
the changes are propagated to other users across the network.
More concretely, RTE systems can be defined by the following characteristics \cite{Sun1998}:
(i) the user responsible for a change is able to observe it \textit{real-time};
(ii) collaboration is allowed to take place in a \textit{distributed} way across
different networks \textemdash which involves a layer of non-deterministic communication
latency; (iii) the system allows for \textit{unconstrained}, non-locking edit operations
\textemdash i.e. any user can edit the shared data at any time.

Thus, one major technical puzzle a RTE system has to solve consists of the fact that changes
are propragated across networks with different communication latencies and therefore do not
reach all end points in the same order. For example, given
the string ``efecte'' and concurrent edit operations \textit{Ins(2, ``f'')} \textemdash i.e.
insert character ``f'' at position 2 \textemdash and \textit{Del(5)} \textemdash i.e.
delete character ``e'' at position 5 \textemdash, the spell-corrected string ``effect''
can only be obtained if \textit{Del(5)} is executed before \textit{Ins(2, ``f'')}.

A common solution to consistency maintenance is Operational Transformation
(OT) \cite{SuClarence}, a technique that consists in a rule-based transformation
of edit operations which amends the position of an edit operation against a previous edit.
In the previous example, the edit sequence \textit{[Ins(2, ``f''), Del(5)]} should be
transformed as follows:
\begin{equation*}
  \textit{OT([Ins(2, ``f''), Del(5)])} \Rightarrow \textit{[Ins(2, ``f''), Del(6)]}
\end{equation*}
to ensure consistency with respect to the alternative edit sequence
\textit{[Del(5), Ins(2, ``f'')]}.

%% Three types of divergences:
%% Divergence
%% Causality Violation
%% Intention Violation

Finally, OT-based RTE systems work with linear data-structures, typically lists of
characters, but it is possible to work with list of elements at other levels, such as words,
paragraphs or XML nodes \cite{Imine2009,SuClarence}. For instance, in a RTE annotation application,
one might want to consider labels (annotation keys and values) instead of characters as the
atomic targets of edit operations.

%% Limitations...?

\subsection{Version Control System}\label{subsec:vcs}

VCSs have a large tradition originating in the 1970s as groupware for enabling collaborative
software development and more recently have become widely adopted as a result of
integration with a social network environment (GitHub, BitBucket).
Current VCSs follow the copy-modify-merge model, also known as ``optimistic'', to
collaborative document editing, in opposition to the ``pessimistic'' lock-modify-unlock model
in use among early VCSs such as SCCS, RCS or CVS \cite{Loeliger2012}.
In order to guarantee consistency maintenence the lock-modify-model forces users to first
acquire a lock for the target file and release it once changes are finished. However, 
the lock model proved to be too restrictive in practice.

The optimistic flavour of VCS represents an asynchronous collaboration model based on a
two-step procedure of local editing and merging with the shared document.

In principle, this two-step procedure resembles the one described in section \ref{subsec:rte}
for RTE systems. However, they essentially differ in that in VCS local changes are not merged
immediately (synchronously) into the shared document (as opposed to edit prograpagation
within the RTE framework); instead they are ``commited'' at a later stage
\textemdash i.e. asynchronously.
By separating edit and merge logic real-time responsiveness is
sacrified but in exchange two crucial features are gained: (i) the concept of a
\textit{working copy}, (ii) the fact that edits are meaningful \textemdash in the sense
that succesful merges can be deemed as intended \textit{versions} of the document.
We now turn to these aspects in more depth.

\paragraph*{Working copy}
A working copy is a local copy of the shared document that allows users
to modifying the document (in their local environment) despite parallel work by other users
\cite{Collins-Sussman}. The concept of working copy has the potential for VCS implementations
that are completely distributed \textemdash known as DVCS (Distributed Version Control Systems). %citation needed?
The general collaborative workflow with working copies can be described as folllows:
When a user wants to update the shared data to reflect the local changes in her working copy
(commits), a ``push'' operation is issued. Similarly, other users can update their local
working copies to reflect the current state of the shared data \textemdash an operation
known as ``pull'' in some implementations of this model.
In both cases a conflict between the local and shared copies may arise if (i) the shared
copy was successfully updated in a particular way by another user before the local changes
are pushed, (ii) or there have been local changes incompatible with the shared copy before
the latter is pulled.
VCS provides a merging mechanism to solve these potential version divergences known as
three-way merging, which involves all three document copies (both conflicting copies and their
shared ancestor) \cite{Altmanninger2009}. In general, VCSs provide automatic merging in cases
of ``syntactically'' unambiguous and it is commonly resorted to manual merging in cases
of ``semantic'' ambiguity, in which conflicting edit intentions may be present.%citation needed.

A crucial aspect of working copies is that they allow for more complex and flexible workflows
than the previously depicted case of a single working copy. In particular, users may create
multiple working copies, each of which can focus on a certain aspect of the project, and
delay merging to a later stage.
By this token, the workflow can be seen as a tree of branches (working copies
representing parallel versions of the shared data) that are being splitted and merged with
the trunk (the shared data).

\paragraph*{Versioning}
Versioning is a powerful feature for backtracking changes and restoring project state to a
previous one. Versions can be generally defined as the state of a document in a given time point.
In the context of VCSs, a version refers to the project state resulting from a successful merge.
Moreover, the asynchronous nature of VCS commits allows attaching metadata to commits before
merging. This is in opposition to RTE document updates that are triggered automatically
after any modification and propagated to other users.

Besides enabling responsability traceability, version metadata provide grounds to build useful
collaborative features with respect to the following two aspects: (i) edits to a document become
meaningful allowing for the specification of the motivations behind the edit as well as to link
to metadata in previous versions in a structured way \textemdash while structuring of the metadata
is not enforced by default, it can be easily introduced by following particular internal project
conventions; (ii) the resulting tree-shaped\footnote{
  Strictly speaking, the graph structure resulting from the collaborative model of copy-edit-merge
  with branching does not correspond to a tree due to the presence of merges but rather to a DAG
  (directed acyclic graph).
} edit history can be easily traversed back, choosing which particular development branch should be
followed, and therefore enabling more specific state restoration.

Finally, as a consequence of the concept of working copies, project state can be again restored
to a certain version independently of concurrent actions by other users, allowing for parallel
versions of the same document. In fact, another popular feature of VCS implementations
is ``forking'' (creating a parallel version of the project, which may eventually be merged
again with the original trunk).

%% \begin{verbatim}
%%   {"word": "the", "pos": "AT"} \\ Brown Tagset
%%   {"word": "the", "pos": "DT"} \\ PTB Tagset
%% \end{verbatim}

%%                      _branch: brown->PTB     _ merge
%%                     /                         \
%% - user1 uses brown - continue adding more PTB  --
%%                                                   - merge ----
%% - user2 uses PTB   - continue adding more PTB  --
%%                     
%% --
%% - user1 starts using PTB
%% - user3 branches to correct previous annotations

%% Commits are a further layer of asynchrony between local changes and local branches?.
%% Versioning systems
%% Collabode (bringing RTE to VCS)
%% Deep hypertext with embedded revision control (bringing VCS to RTE on the web)

\subsection{Advantages of Asynchronous Collaboration for Corpus Annotations/Hybrid solution}
\label{subsec:hybrid}

Having outlined RTE and VCS approaches to document collaborative editing, we now proceed to discuss
their suitability and aptness to handling synchronous corpus metadata creation.

We note that RTE is conceptually simpler (although its implementation is non trivial and 
constitutes a topic of current research): from a user perspective, little overhead is added to
the actual annotation process.
This conceptual simplicity \textemdash together with high GUI responsiveness \textemdash, provides
a better user experience, which explains why RTEs are more widely used by average users.
This also means that the synchronization logic is done in the background and eludes user control.
VCS is, in comparison, more involved and requires grounded knowdlege of the underlying mechanism,
but it also offers a more powerful collaboration model that supports complex collaborative
workflows and better version metadata handling.
In our view, \ref{sec:intro} the asynchronous copy-modify-merge approach better fit the  
contribute to an incipient trend of expansion of VCSs from the Software Development community
to other areas of group-based data creation \textemdash in particular, scientific research\footnote{
  As commented by GitHub CEO in a recent interview \cite{Begel2013}.
}.

Furthermore, we consider that a significant increase in research quality can be obtained as
a result of rich and structured versioning metadata. Easier metadata export capabilities leads
to better reproducibility and aggregation. In many cases, annotation metadata can be utilized in
order to better understand the phenomenon at stake ().
%research using research metadata to improve systems.
%% For instance in 

Researchers working with large sequential datasets (such as text corpora) will most often find
themselves analyzing chunks in the absence of concurrent work by other peers\footnote{
  This is not to say that the analysis can be carried out outside the scope of a collaborative system
  since updates still have to be brought to the attention of other users and documented.
}, meaning that updates will result in conflicts with lower probability than in other setups.
We can take advantage of this fact in order to alleviate some aspects 
simulate more RT interaction.
As we will see in the next section, we also improve    reducing the complexity of actual
implementations of VCSs by including only a minimal subset of neccessary features.

\section{Case Study?}\label{sec:case}

\subsection{Architecture}
Decoupling of the annotation application and the search engine. Enforce modularity.
Ideally, one should be able to swap backend search engines in a seamlessly way.

\subsubsection{Concoordancer}


\subsubsection{Annotation Database}
Data Model, refer to RelANNIS.
- Versioning,
- Branching, Centralized around (given that our app is tightly connected to a corpus query backend)
a search engine which not only can be stored given to high memory usage and technical complexity.
Always open to client implementation.

\subsubsection{Web-based Client}
Enabling real time interaction between users.

\subsection{Use Cases}
How our application addresses the problems specified in Section \ref{sec:intro}

\subsection{Limitations}
By decoupling annotation app from the backend search engine, we lose the power to incorporate
the body of on-going annotations.
While we acknowdlege this to be an important limitation of our current architecture, it need
also be pointed out that current corpus search engines are built around static corpus
representations, thus not allowing for incremental indexing of new metadata.

\section{Future Research}\label{sec:future}
Integration with other tools, being agnostic with respect to which search engine is used,
providing a simple API (only requirements are that the backend allows to retrieve data via
a communication protocol such as HTTP or RPC and that tokens are given an id corresponding
to their position in the corpus.

\section{Conclusions}\label{sec:conclusion}

\section{Acknowledgements}

%% \section{References}
% The following two commands are all you need in the
% initial runs of your .tex file to
% produce the bibliography for the citations in your paper.
\bibliographystyle{abbrv}
\bibliography{/home/enrique/Documents/texts/bib/library}
% You must have a proper ".bib" file
%  and remember to run:
% latex bibtex latex latex
% to resolve all references
%
% ACM needs 'a single self-contained file'!
%

%\balancecolumns % GM June 2007
% That's all folks!
\end{document}
